\documentclass[notitlepage, amsmath, amssymb, aps]{revtex4-1}

\begin{document}

\title{BAR with 3 ensembles}

\author{Dong Zhou}
\date{\today}

\begin{abstract}
In the same spirit of the original BAR paper, the idea is to find the minimum variance solution
of the free energy difference estimator with respect to the weight functions.
\end{abstract}

\maketitle

\section{introduction}

The quantity of interest is the free energy difference between the two end points:

\begin{align}
\Delta A_{02} = \Delta A_{01} + \Delta A_{12}
\end{align}
where $\Delta A_{ij} \equiv A_i - A_j$ is the difference between two ensembles.

For now, let's introduce two weight functions $w_{01}$ and $w_{12}$ for the neighboring ensembles.
(We could also introduce $w_{02}$ and combine different estimators for $\Delta A_{02}$. See section~\ref{sec:combine}.)
Following the same derivation of BAR paper, we have

\begin{align}
\Delta A_{02} = \log\left<w_{01} e^{-u_1}\right>_0 - \log\left<w_{01}e^{-u_0}\right>_1
              + \log\left<w_{12} e^{-u_2}\right>_1 - \log\left<w_{12}e^{-u_1}\right>_2
\end{align}

Note that the second and third terms are not statistically independent.
If we ignore their covariance, then the variance of $\Delta A_{02}$ would simply be the sum of
the variances of $\Delta A_{01}$ and $\Delta A_{12}$, and we can use the BAR result directly for the weight functions.
The existence of an extra covariance term tells us that applying the 2-ensemble BAR formula for 01 and 12 separately is not optimal.

\section{covariance term}
I will use the following two formula without proof

\begin{align}
Var[f(X)-g(X)] &= Var[f(X)] + Var[g(X)] + 2Cov[f(X), g(X)] \\
Cov[f(X), g(X)] &= f'(\left<X\right>)g'(\left<X\right>) Var[X]
\end{align}
where $X$ is a random variable, and $\left<X\right>$ is its expectation value with respect to an ensemble.
The first one is trivial. The second one can be easily derived by expansion to second order.

Recall in our case

\begin{align}
f(X) =& \log\left<w_{12} e^{-u_2}\right>_1 \\
g(X) =& \log\left<w_{01}e^{-u_0}\right>_1
\end{align}

Similar to the variance derivation in my blog post, the covariance is given by

\begin{align}
Cov[f(X), g(X)] = \frac{1}{N_1}\sqrt{\left(\frac{\left<w_{12}^2 e^{-2u_2}\right>_1 }{\left<w_{12} e^{-u_2}\right>_1^2 } - 1\right)\left(\frac{\left<w_{01}^2e^{-2u_0}\right>_1}{\left<w_{01}e^{-u_0}\right>_1^2} - 1\right)}
\end{align}


\section{variance optimization}

Collecting all 4 variance terms and the covariance term, we have

\begin{align*}
Var[\Delta A_{02}] =& \frac{\left<w_{01}|C_{01}+C_{10}|w_{01}\right>}{\left<w_{01}|b_{01}\right>\left<b_{01}|w_{01}\right>}
            + \frac{\left< w_{12}|C_{12} + C_{21} |w_{12} \right>}{\left<w_{12}|b_{12} \right>\left<b_{12}|w_{12} \right>} \\
            &+ 2\sqrt{\left(\frac{\left<w_{01}|C_{10}|w_{01}\right>}{\left<w_{01}|b_{01}\right>\left<b_{01}|w_{01}\right>} -\frac{1}{N_1}\right)
                     \left(\frac{\left< w_{12}|C_{12} |w_{12} \right>}{\left<w_{12}|b_{12} \right>\left<b_{12}|w_{12} \right>} -\frac{1}{N_1}\right)}
            - \frac{2}{N_1} - \frac{1}{N_0} - \frac{1}{N_2}
\end{align*}
where the bra-ket notations are
\begin{align}
\left|b_{ij} \right> &= e^{-u_i - u_j} \\
\left|w_{ij} \right> &= w_{ij} \\
C_{ij} &= \frac{Q_i}{N_i} e^{-2u_j - u_i}
\end{align}

To further simplify the notation, let's define
\begin{align}
    \gamma_{ij} &\equiv \frac{\left<w_{ij}|C_{ij}|w_{ij}\right>}{\left<w_{ij}|B_{ij}\right>\left<B_{ij}|w_{ij}\right>} \\
    w_{ij} &= w_{ji}
\end{align}

Then we have
\begin{align}
    Var[\Delta A_{02}] = \gamma_{01} + \gamma_{10} + \gamma_{12} + \gamma_{21}
                        + 2\sqrt{\left(\gamma_{10} - \frac{1}{N_1}\right)\left(\gamma_{12} - \frac{1}{N_1}\right)}
                        - \frac{2}{N_1} - \frac{1}{N_0} - \frac{1}{N_2}
    \label{eq:var_02}
\end{align}

Setting its first variation to 0 with respect to $\left<w_{01}\right|$ and $\left<w_{12}\right|$, we get two secular equations.
One of them is

\begin{align}
    \frac{1}{\gamma_{01}+\gamma_{10}\left(1+\sqrt{\frac{\gamma_{12}-\frac{1}{N_1}}{\gamma_{10}-\frac{1}{N_1}}}\right)} \left|w_{01}\right> = \left(C_{01}+C_{10}\left(1+\sqrt{\frac{\gamma_{12}-\frac{1}{N_1}}{\gamma_{10}-\frac{1}{N_1}}}\right) \right)^{-1} \left|b_{01}\right>\left<b_{01}|w_{01}\right> 
\end{align}
The other one has the same form.
In the derivation we have used
\begin{align}
    \frac{\partial\gamma_{ij}}{\partial \left<w_{ij}\right|} = \frac{C_{ij} - \gamma_{ij}\left|b_{ij}\right>\left<b_{ij}\right|}{\left<w_{ij}|b_{ij}\right>\left<b_{ij}|w_{ij} \right>}\left|w_{ij}\right>
\end{align}

Because $\left|b_{ij}\right>\left<b_{ij}\right|$ is a projection, these secula equations have unique solutions.
For example

\begin{align}
    \left|w_{01}\right> = & \left(C_{01} + C_{10}\left(1+ \sqrt{\frac{\gamma_{12}-\frac{1}{N_1}}{\gamma_{10}-\frac{1}{N_1}}}\right) \right)^{-1}\left|b_{01}\right> \label{eq:w01} \\
    \frac{1}{\gamma_{01} + \gamma_{10}\left(1+\sqrt{\frac{\gamma_{12}-\frac{1}{N_1}}{\gamma_{10}-\frac{1}{N_1}}} \right)} = & \left<b_{01}\right| \left(C_{01} + C_{10}\left(1+\sqrt{\frac{\gamma_{12}-\frac{1}{N_1}}{\gamma_{10}-\frac{1}{N_1}}} \right) \right)^{-1}\left|b_{01}\right>
\end{align}

Note that this solution of $w_{ij}$ differs from the 2-ensemble BAR result by a square root term.
And if we simply use the 2-ensemble BAR result of $w_{ij}$,
the overall variance is not simply the sum of individual variances:
there is an extra covariance contribution in $\Delta A_{02}$ that is not currently accounted for, as seen in Eq.~\ref{eq:var_02}.

There are a few things we can do at this point:

\begin{itemize}
    \item Seek exact analytical solution. Maybe use ansatz solutions, e.g., summation form, product form; or introduce another variable for the square root term, and see what equation it satisfies.
    \item Develop approximate solution. For example, use perturbation theory with the 2-ensemble BAR result as zeroth order; or introduce some simplifying assumptions.
    \item Seek numerical solution. For example we can probably use the 2-ensemble BAR result as starting point for an iterative solver.
    \item Find the physical meaning of $\gamma_{ij}$ and range of its value. To start with, we can use some special form of $w_{ij}$.
    \item Find how the other multi-ensemble methods link to this solution. What are their simplifying assumptions?
\end{itemize}

\section{extension to $N$ ensembles}

We are interested in the free energy difference between the two end ensembles.
There are different ways to form this estimator (see next session),
the simplest is to use only estimators of neighboring ensembles, i.e.,

\begin{align}
    \Delta A_{0N} = \sum_{i=0}^{N-1}\Delta A_{i,i+1}.
\end{align}

The derivation of its variance is similar to the 3-ensemble case:
there is one covariance terms for each internal ensembles, i.e.,

\begin{align}
    Var[\Delta A_{0N}] = \sum_{i=0}^{N-1}Var[\Delta A_{i,i+1}] 
    + 2\sum_{i=1}^{N-2}Cov[\log\left<w_{i-1,i}e^{-u_{i-1}}\right>_i, \log\left<w_{i, i+1}e^{-u_{i+1}}\right>_i].
\end{align}

And the corresponding weight functions have the same form of Eq.~\ref{eq:w01},
simply replacing $1$ with $i$, $0$ with $i-1$, and $2$ with $i+1$.


\section{combining different estimators} \label{sec:combine}

With $N>2$, there exist multiple estimators for the free energy difference of the two end ensembles.
We can combine them into new estimators.
Take $N=3$ for example,

\begin{align}
    \Delta A = \alpha (\Delta A_{01} + \Delta A_{12}) + (1-\alpha)\Delta A_{02}
\end{align}
where $\alpha$ is to be determined.

In this case, there is one covariance term for each ensemble, both internal and end ones.


\end{document}
